\documentclass[11pt]{article}

% Included packages:
\usepackage{graphicx}
\graphicspath{ {./images/} }

% Document definitions:
\setlength{\oddsidemargin}{1in}
\setlength{\evensidemargin}{1in}
\setlength{\marginparwidth}{0.0in}
\setlength{\topmargin}{0.5in}
\setlength{\headheight}{0in}
\setlength{\headsep}{0in}
\setlength{\textwidth}{4.5in}
\setlength{\textheight}{7.5in}
\setlength{\parindent}{0.25in}
\setlength{\parskip}{0.0in}
\newfont{\hvb}{cmssbx10}
\newfont{\hv}{cmss10}
\newfont{\tir}{cmr10}
\setlength{\baselineskip}{10pt}

	

%%%%%%%%%%%%%%%%%%%%%%
%%% BEGIN DOCUMENT %%%
%%%%%%%%%%%%%%%%%%%%%%
\begin{document}
\begin{center}
% Title And Authors:
{\footnotesize{\hvb
Optimizing Settling Time for Swarm UAV Formation Control In the Presence of Obstacles\\
}}

\vspace{10pt}
{\footnotesize
{\hvb Raj Shah}, {\hv M.S. Student (Mechanical)} \\
{\hvb Supreet Kurdekar}, {\hv M.S. Student (Mechanical)}\\
{\hvb Chris Gnam}, {\hv Ph.D. Student (Aerospace)} \\
{\hv
Department of Mechanical and Aerospace Engineering
University at Buffalo
Buffalo, New York, 14261
}}
\end{center}

UAVs are used in a wide range of applications such as land surveys, search and rescue operations, remote sensing, disaster monitoring to security, defence and localization applications. A group of UAVs greatly enhances the performance and robustness in fulfilling difficult tasks. Multiple agents increase the chances of completing a mission as other agents can continue the mission even if one or more agents fail. Moreover, a group can acquire more information as it can carry more sensors. However, a variety of applications such as exploration, land survey, search and rescue missions, forest fire detection and surveillance, mapping and deployment of troops require agents to maintain the desired formation and therefore it is essential for a swarm to have precise formation control with a quick response to environmental disturbances such as obstacles and minimum settling time. In this report, we attempt to optimize settling time in formation control of UAVs swarm navigation in the presence of obstacles.

“What problem are you solving?”

Single simple obstacle (cylinder, square, etc.).  With 4 agents.  Start with square formation.  Obstacle size will be k*L where L is the width of the formation.  We’ll do UAV drones.  Initial velocity is non-zero and average coasting speed of a UAV drone (TBD).  We’ll apply appropriate constraints on the dynamics based on physical limitations of the drones.

Our system converges if the consecutive changes are less than 2% of the initial length (potentially formalize with a formal definition of settling time).  Way of parameterizing this is TBD.  It would be calculated for both the initial formation and the final formation.  Could be:
Sum of all distances between points (Before and after)
Square of difference (like a “least-squares” approach)
RMS, etc.

“interaction distance”, and if the obstacle is within that distance, it begins to maneuver.

We are going around an obstacle.  Don't care about the final shape.  Just care about the d0 (before and after must match within tolerance)

No limitations on sensor speed, and perfect information of all other drones.

“Why is it important?”
Multi-agent swarms offer an opportunity for robust distributed control.

Decreased settling time in response to obstacle avoidance means less interruption to nominal operations, which can be important in scenarios where the swarm is used for data collection.  The quality of data collected can be decreased when the swarm has split up.
Talk about the importance of swarms
Talk about swarms operating in the presence of obstacles
Talk about the importance of settling time optimization for swarms

“Literature Survey:”

In Flocks, Herds, and Schools: A Distributed Behavioral Model, Reynolds created a decentralised flock motion scheme to simulate the motion of an entire flock if agents without defining individual paths for all the agents. He accomplishes this by modelling the entire motion as an outcome of each agent following three simple rules. Avoidance: avoid collisions with nearby flockmates.  Velocity Matching: attempt to match velocity with nearby flockmates.  Flock Centering: attempt to stay close to nearby flockmates. These simple rules have been formulated in many different ways to create new methods of formation control.
In Stable Flocking of Mobile Agents (H.G. Tanner), a simple methodology for stable flocking configurations was introduced, utilizing potential functions for mediating inter-agent formation keeping maneuvers.  While this method proved powerful for both acquiring and maintaining a stable flocking formation, it did not consider obstacles in its formulation.

In Virtual Leaders, Artificial Potentials and Coordinated Control of Groups (N.E. Leonard and E. Fiorelli), the concept of a “virtual leader” was introduced. A virtual leader is an artificial point that maintains formation shape by attracting individual agents towards itself. This also proves advantageous as the motion of the entire swarm can be affected by controlling the virtual leader. 

 In Unmanned aerial vehicle swarm control using potential functions and sliding mode control. Artificial potential functions are defined. These potentials are functions of inter agent distance. A repulsive potential is created around the obstacle. This too is a function of the distance between the agent and the obstacle. Finally the potential function for any agent is the sum of potentials due to interagent distance and obstacle potential. Typically each agent has a potential applied to it and the control input for that agent uses a steepest gradient policy in the control input to minimise the potential function value. Reaching the minima of this potential function means that the agent has achieved required distance from all the neighbouring  agents. 
	This work utilized a sliding mode control strategy to incorporate system uncertainty into the control input in order to make the control strategy robust to real world implementations.

Finally A Lightweight Formation Control Methodology for a Swarm of Non-Holonomic Vehicles (G. H. Elkaim and R. J. Kelbley) presented a similar methodology for maintaining stable flocking motion, however it included additional potential fields to mediate interactions between the agents a virtual leader, and obstacles in the environment.  


We will be minimizing the settling time, by defining an objective function for the settling time, with respect to the control gains and the reaction distance from the obstacle (potential field size), which represent our design variables.  There are several constraints on this problem, however they are fairly straightforward.  First and foremost, the vehicles must not collide with either each other or with the obstacle.  In addition, the vehicles will have a maximum allowable acceleration (applied actuation), and will have a maximum allowable traveling velocity.

The simulation for this project will primarily be done in MATLAB.  Preliminary work has already been started implementing the papers we previously discussed so we can develop a better working knowledge of the problem.  Currently the dynamics are modeled as a simple 2-d point mass particles subject to input accelerations.  As the project progresses, we may switch to a higher fidelity model, depending on what is available.

The expected outcome of this project is to determine the optimum control gains, and reaction distance (as mediated by the obstacle potential functions), so as to minimize the settling time of the flock responding to an obstacle.

To go about this, the tasks of the project will be evenly distributed among the group members.  All three members will be involved with both the formulation of the optimization problem, as well as the writing of the final report.  Though each member does have their own unique responsibilities.  The git repository for the project, as well as the main architecture of the code, will be maintained by Chris, while Raj will focus more on the implementation of simulation models and optimization routines.  Supreet will be tasked primarily with designing the overall problem to be optimized.

%\includegraphics[width=\textwidth]{•}
\end{document}

